\chapter{Research Plan}
	
	In this chapter an outline of the work to be carried out is presented. Due to
	the embryonic stage of the research, later work is presented with less detail.
	Figure \ref{fig:plan-gantt} shows a Gantt chart containing the key stages of
	the project which are outlined in chronological order in the sections below.
	
	\begin{figure}[b!]
		\center
		\input{figures/plan-gantt}
		
		\caption[Gantt chart of proposed research plan]{Gantt chart of proposed
		research plan. Arrows indicate dependencies, thick lines indicate slack.
		Note non-linear scale.}
		\label{fig:plan-gantt}
	\end{figure}
	
	
	\section{SpiNNaker Modelling}
		
		% In order to experiment with the way new interconnection options affect
		% SpiNNaker an accurate model is required. Will work with Mikel and Javier to
		% extend the SpiNNaker simulator used in the original paper to include models
		% of inter-board links.
		%
		% How long will this take?
		
		% TODO: What journal?
		
		As part of the ongoing SpiNNaker project a study is being carried out to
		compare how network simulations compare with the real hardware. As part of
		this work, researchers have built an FPGA accelerated model of SpiNNaker's
		interconnection network. The work hopes to verify the accuracy of the model
		against both a traditional software model and also the final silicon
		implementation. The results of this work are intended for publication in the
		Journal of TODO by the end of September 2013.
		
		My participation in this project will be to develop the software simulation
		to model and evaluate the network implemented in SpiNNaker. Once built, this
		simulator will be valuable for my own research into alternative network
		topologies.  This section outlines in further detail the motivation and
		requirements for the simulator followed by a discussion of its applications
		for my own work.
		
		\subsection{Software Simulator Limitations}
		
			The prototype software model is built on the INSEE simulator
			\cite{navaridas11insee} which is designed to simulate a wide variety of
			networks. Unfortunately, INSEE's router model is different from that found
			in SpiNNaker and the FPGA model. The differences in their behaviour are
			outlined below.
			
			% TODO: elaborate
			
			SpiNNaker makes use of store-and-forward routing where a message must be
			fully received before it can be routed to the next point in its path. By
			contrast, INSEE is designed to support cut-through routing where once the
			first part of a message has been received it may immediately begin the
			routing process.
			
			The other key difference between INSEE and the actual SpiNNaker
			architecture is the way packets from incoming links are arbitrated. Figure
			\ref{fig:arbitration} shows the arbitration schemes for INSEE and
			SpiNNaker. INSEE uses a simple round-robin arbitration scheme while
			SpiNNaker (and the FPGA model) use a tree of arbitrators\footnote{The
			bandwidth available at each level of the arbitration tree scales with the
			maximum input bandwidth for each level.}.
			
			\begin{figure}
				\begin{subfigure}[t]{0.50\textwidth}
					\center
					\begin{tikzpicture}[thick]
	
	\node (arbitrator)
		[draw,minimum height=6cm]
		{
			\tikz[minimum height=0cm]
				\node[rotate=90]{RR};
		};
	
	\newcommand{\inputline}[2]{
		\draw [<-]
		      ($(arbitrator.north west) !#1! (arbitrator.south west)$) -- ++(-0.5,0)
		      node [left] {#2}
		      ;
	}
	\inputline{0.1}{N}
	\inputline{0.2}{S}
	\inputline{0.3}{E}
	\inputline{0.4}{W}
	\inputline{0.5}{NW}
	\inputline{0.6}{SE}
	\inputline{0.7}{P1}
	\inputline{0.8}{\ldots}
	\inputline{0.9}{P18}
	
	\draw [->]
	      ($(arbitrator.north east) !0.5! (arbitrator.south east)$) -- ++(0.5,0)
	      node [right] {Router}
	      ;
	
\end{tikzpicture}

					
					\caption{INSEE}
					\label{fig:arbitrationINSEE}
				\end{subfigure}
				\begin{subfigure}[t]{0.50\textwidth}
					\center
					\input{figures/arbitrationSpiNNaker}
					
					\caption{SpiNNaker}
					\label{fig:arbitrationSpiNNaker}
				\end{subfigure}
				
				\caption[Incoming packet arbitration schemes in INSEE and
				SpiNNaker]{Incoming packet arbitration schemes in INSEE and SpiNNaker.
				Boxes marked `RR' are round-robin arbitrators, P1-P18 are connections to
				local processor cores.}
				\label{fig:arbitration}
			\end{figure}
			
			This difference here extends beyond the order in which contesting requests
			will be serviced. In INSEE, each input has a buffer associated with it
			from which the router will extract messages and forward them to the input
			buffer of the next node, one per simulated cycle. In the SpiNNaker design,
			messages are buffered at each level of the tree and eventually placed in a
			pipeline within the router (equivalent to further buffering) and an output
			buffer. The interaction of all these buffers is not modelled by INSEE and
			thus the results produced are less well matched with the actual SpiNNaker
			system.
			
			% Part of the aim of the work is to produce a comparison between this
			% software model, an FPGA model and a prototype system. Successful
			% comparison of the three will allow higher certainty in results obtained
			% from other work (namely the interconnect experiments to follow).
		
		\subsection{Simulator Improvement Plan}
			
			Given the limitations of INSEE mentioned in the previous section, two
			possible approaches must be considered. Either INSEE must be modified to
			incorporate a more realistic model of the router or an alternative
			simulator must be used.  One important factor in the decision is the
			`ramp-up' time required to gain familiarity with the INSEE code-base
			compared with the time required to develop or extend an alternative
			simulator. The other factor is the utility of the simulator for my own
			research.
			
			Since INSEE is an established tool which has been used in similar work it
			is potentially a strong choice. In order to determine its suitability for
			this work and my own research a small amount of time will be initially
			spent analysing its design. If it is found to be suitable development of
			an extended version will commence.
			
			A possible alternative to INSEE is the simulator developed during the
			preliminary interconnect study in \S\ref{sec:interconnect-modelling}.
			Like INSEE, it has been already been used to simulate the SpiNNaker
			Interconnect and so configurations exist for SpiNNaker like machines.
			Because of the author's familiarity with the tool, extending the router
			model should be straight-forward.
	
	
	\section{Small-World Network Experiments}
		
		With the extended and proven simulator developed, the next stage of my
		research will be to use it to model the behaviour of small-world style
		networks with more realistic traffic and wiring constraints.
		
		The work done in \S\ref{sec:small-world-super-computers} measured average
		shortest-path length in the networks examined. This measure corresponds to
		the average path length for uniform-random traffic in a real network with
		the same topology. In brain simulation systems such as spinnaker, much of
		the traffic is local and so uniform-random traffic is not representative.
	
	
	\section{Topology Comparison}
		
		Look at a whole load of popular topologies and possibly look at having two
		distinct topologies. Saying this, will this end like the I/O space in x86?
	
	\section{Place and Routeability}
		
		Given more complex architecture and the massive NP-complete task of place
		and route, what can be done? Work will look at possible schemes and
		architectures which help with that.
	
	\section{Effects of Multicast}
		
		Multicast networks are unusual in many respects and complicate things
		somewhat. The model should be further extended to cope with multicast
		traffic and further work carried out in this aim.
	
	\section{Interconnect Technology Evaluation}
		
		Can't use Silistix any more, HSS is nice but should we be using it on-chip?
	
	\section{`SpiNNaker 2' Architecture}
		
		It is likely that a new version of SpiNNaker will be designed with some
		funding coming in soon. Based on the research into the interconnect here,
		would an alternative topology be appropriate?
		
		\subsection{Types of Transmission}
			
			Switching away from Silistix (they went bye-bye!), alternative
			technologies, e.g. high-speed serial, might be used to link chips
			together. Such technologies may have very different cost structures and
			call for new network topologies. Given we currently have concentrated
			multiplexed links, why not have the same for stuff lower down?
		
		\subsection{Multiple Networks}
			
			SpiNNaker's interconnect is obviously heavily targeted at neural
			simulation but at the expense of other activities. For example, set-up and
			management are a bit of a pain. Also, other algorithms don't suit it so
			well. It is possible SpiNNaker will benefit from having a second network
			for these purposes?
			
			What about time-division-multiplexing two different protocols onto one
			high speed serial link?

