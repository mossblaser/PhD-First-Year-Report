\chapter{Introduction}
	
	\section{Parallel Computing}
		
		The race to drive up the speed of individual processor cores ground to a
		halt in the mid 2000s leaving designers pushed to find a use for the vast
		number of transistors made available by Moore's law. In the years that
		followed we've seen multi-core processors appearing in everything from
		mobile phones to large super computers.
			
		\subsection{Programming}
		
		\subsection{Architecture}
	
	\section{Computational Problems}
		
		\subsection{TODO: More Examples}
		
		\subsection{Fluid Dynamics}
		
		\subsection{Brain Simulation}
			
			% TODO: Cite overview of computational neuroscience?
			
			The brain is an extremely powerful computer about which little is
			understood. The field of computational neuroscience hopes to bring
			understanding of the computational abilities and mechanisms of the
			brain.
			
			One approach to this problem is to try and produce simulations of models
			of the brain in order to understand and study their behaviour. These
			approaches often take the form of simulated populations of neurons, one
			of the basic building-blocks of the brain. Such models typically consist
			of a set of neurons, each connected to many other neurons.  Unlike
			digital circuits where each individual component connects to only a few
			or even one other component, neurons tend to be connected to hundreds or
			thousands of other neurons.
			
			Simulations of the brain therefore present a computational challenge as
			large amounts of communication must take place between all the neurons
			in the simulated system. Various architectures have been designed to
			solve this task and a selection are presented in Chapter
			\ref{chap:background}.
	
	\section{Architecture}
		
		\subsection{Topology}
		
		\subsection{Routing}
		
		\subsection{Multicast}

