\chapter{Conclusion}
	
	The challenge of understanding the brain has led to the construction of vast
	simulated neural models. Over time, models have increased in size, complexity
	and realism. Though the Blue Brain project has simulated detailed models of
	small systems of hundreds of thousands of neurons using an `off-the-shelf'
	super-computer, such models pale in scale to the human brain with its 100
	billion neurons.  Large networks of simple neurons, such as Spaun, have shown
	that complex and biologically realistic behaviour can be generated by simple
	models of neural behaviour.
	
	Conventional super-computers feature large amounts of computational resource
	interconnected by a topology designed for moving large blocks of data at high
	bandwidths.  Large scale simulations of simple neuron models do not fit the
	expectations of such machines as they feature only a limited computational
	need instead requiring the transmission of large numbers of extremely small or
	empty messages with a high sensitivity to latency. This mismatch has led to
	the development of special-purpose architectures for brain simulation.
	
	Many brain simulation architectures have been built on custom analogue and
	digital circuits which support certain models of neuron. Such devices have
	achieved extremely efficient simulations of large numbers of neurons, for
	example the Neurogrid architecture is able to simulate one million neurons
	while consuming orders of magnitude less power than the super-computer used
	for Blue Brain. Meanwhile the BrainScaleS project has produced a wafer-scale
	system which can simulate neural models up to tens of thousands of times
	faster than their biological counterparts.
	
	Though typically fast and power efficient, the lack the flexibility to run
	other neural models than those intended for them. As well as being
	computationally limited, their topologies also often place constraints on
	their scalability. For example, the Neurogrid topology fails to exploit the
	on-chip parallelism available in their architecture allowing only one neuron
	to spike at a given time. Other systems, such as BrainScaleS, feature
	topologies whose dimensions are limited to that which can be manufactured in a
	single silicon wafer or chip.
	
	These limitations are being tackled by the SpiNNaker system which combines
	large numbers of small-low power general purpose processors in a scalable
	network with the aim of simulating networks of one billion neurons in
	biological real time. The SpiNNaker topology forms a regular 2D toroid built
	out of boards containing 48 chips connected together by delay-insensitive
	asynchronous parallel links. The boards are then connected using high-speed
	serial links to reduce the number of wires required to connect the boards
	together.
	
	A simulation of the non-homogeneous topology used by SpiNNaker was analysed to
	test the effects on the network latency. Though an 80\% increase in latency
	was measured, this is not expected to significantly affect the system's
	real-time performance. Since the expected median latency increases to only 7.5
	$\mu$s this is still far shorter than the 1 ms time steps used by planned
	neural simulations. Should the period of the time-steps be reduced to improve
	simulation accuracy this problem may become a problem.
	
	A preliminary model of an unconventional semi-random topology was assessed as
	an alternative with results showing improved latencies. Unlike previous work,
	the issue of wiring practicality is addressed and results suggest that gains
	can still be made despite restrictions on wire placement.
	
	The issue of wiring practicality was also investigated for SpiNNaker's
	existing topology. A wiring scheme was developed which ensures that cables
	remain short enough for the high-speed serial communication scheme to function
	when constrained to board placements in standard computer cabinets. In
	addition, the 3,600 wiring instructions was reduced to just 53 making
	construction of large SpiNNaker machines practical.
	
	Work on a new interconnect simulator will allow for the analysis of
	alternative topologies, including the semi-random topology already developed,
	to be tested in a more realistic manner. This simulator will be compared
	against actual SpiNNaker hardware and another, hardware-accelerated simulator
	which will allow the accuracy of the simulator to be assessed.
	
	Based on further refinement of the simulation to account for details such as
	multicast traffic and the issue of resource placement and routing, a new
	architecture will be developed to facilitate the next generation of neural
	simulations. With the continued advance of Moore's law, such an architecture
	may be able to approach models whose size is within an order of magnitude of
	the human brain.
