\chapter{Background}
	\label{chap:background}
	
	Talk about things that have been done by others in the field and try and point
	out any holes and where the pain points are.
	
	\section{Top 500}
		
		Discuss the design of of various Top-500 machines or famous super computers.
	
	\section{Neural Simulators}
		
		Discuss various neural-simulation based approaches, e.g. neuromorphic
		computers, Blue Brain, BrainScaleS, FPGA based things, speak to Franchesco.
	
	\section{SpiNNaker}
		
		The architecture focused on by my research.
		
		\subsection{Neural Networks}
			
			Spiking neural networks. More precise definition of the type of traffic.
		
		\subsection{Real Time}
			
			Real time constraints. Simulations run as time-stepped divisions. Messages
			are assumed to arrive at their destinations instantly because they should
			all arrive by the deadline.
		
		\subsection{Topology}
			
			Due to the neural networks we're using, this is the sort of topology that
			was made. Good for sending short messages to many targets at once. Bad at
			system stuff though, also currently very much 2D unlike the brain. Have a
			hexagonal toroid (picture) which is nice and regular but a pain to wire up.
		
		\subsection{Routing}
			
			Routing is done by table based router. Entries for each turn or fork in a
			packet path. Limited number of entries. Source address based routing (TLA
			to be requested). All routing (currently) static and offline.
			
			\subsubsection{Routing Scheme}
				
				Current routing is very naive dimension order routing. Also highly
				static and doesn't respond to network utilisation. The internet is
				dynamic, maybe this should be too?
			
			\subsubsection{Multicast}
				
				Not much stuff on this at the moment. I am so shit I've not read
				anything about it anyway so that needs to change.
			
		
		\subsection{Hardware Abstractions}
			
			SpiNNaker is made up of 18 core chips on 48-chip boards in 12 card racks
			in 5-rack cabinets in a 10 cabinet system. Woah.
		
		\subsection{Spin-Link}
			
			Boards are connected together via high speed serial links. 8 links per
			board. First gen had two spare connections, new one has ring and one
			spare. What can the spares be used for?
	
	
	\section{High Speed Serial}
		
		Always on, power hungry. Better to have few of these running fast rather
		than many running slow? Where are these used. What do they replace.
